\documentclass[12pt]{article}
\usepackage[portuguese]{babel}
\usepackage[utf8]{inputenc}
\usepackage{indentfirst}
\usepackage{mathtools}
\usepackage{amsmath, amsthm, amssymb, amsfonts}
\usepackage[mathletters]{ucs}
\usepackage{algorithm}
\usepackage{algpseudocode}
\usepackage{tikz}
\usetikzlibrary{matrix}
\usepackage{graphicx}
\usepackage{caption}
\usepackage{subcaption}

\usepackage[margin=3cm]{geometry}
\usepackage{graphicx}

\title{ASA - Relatório 1}
%\subtitle{Análise e Sintese de Algoritmos}
\author{Marisa Roque - 76653}
\date{\today}

\begin{document}
\maketitle


\section*{Introdução}

Foi apresentado um problema que consistia em analisar a forma como se dá a partilha de informação entre pessoas e em como algumas pessoas acabam por formar grupos, supondo que cada pessoa tem um conjunto de pessoas com quem partilha o que recebe. O objectivo deste trabalho é classificar as pessoas em grupos de partilha, de tal forma que quando uma das pessoas do grupo recebe algo, todas as outras desse mesmo grupo também recebem.

Nesse sentido, o programa desenvolvido está preparado para receber informação sobre o modo como se dá a partilha entre pessoas. Esta informação deve conter uma linha inicial com o número de pessoas $N$ e com o respectivo número de partilhas $P$ realizadas por essas mesmas pessoas. Nas restantes linhas de entrada deve vir a informação descriminada entre que pessoas se deu a partilha de informação. As pessoas devem ser identificadas como inteiros entre 1 e $N$, como tal, por exemplo, o facto de vir 1 2 na segunda linha de entrada número 1 quer dizer que a pessoa 1 partilha o que recebe com a pessoa 2.

O programa tem como principal função devolver informação sobre os grupos de partilha, nomeadamente e por ordem de saída, o número de grupos máximos de pessoas que partilham informação, o tamanho do maior grupo máximo de pessoas que partilham informação e, por fim, uma linha com o número de grupos máximos de pessoas que partilham informação apenas dentro do grupo.


\section*{Descrição da solução}

Para chegar a uma solução eficiente passou-se por várias etapas. A primeira foi analisar e perceber detalhadamente os dois exemplos apresentados no enunciado. Para isso foi necessário recorrer à teoria dos grafos, onde são estudadas as estruturas chamadas de grafos, $G(V,E)$, onde $V$ é um conjunto não vazio de objectos denominados vértices e $E$ é um conjunto de pares não ordenados de V, chamado arestas. Daqui rapidamente se partiu para a ideia geral de que se tratavam de grafos dirigidos, dada a natureza do problema, na sua maioria grafos esparsos - aqueles para os quais $|E|$ é muito menor que $|V|^2$ -, e que o que se pretendia era descobrir as componentes fortemente ligadas (SCC) dos mesmos.

Segundo Thomas H. Cormen et al. [1], um componente fortemente ligado de uma grafo orientado $G=(V,E)$ é um conjunto máximo de vértices $C$ $\subseteq$ $V$, tal que, para todo o par de vértices $v$ e $w$ em $C$, temos ao mesmo tempo $v$ $\leadsto$ $w$ e $w$ $\leadsto$ $v$, isto é, os vértices $v$ e $w$ são acessíveis um a partir do outro. Exemplos das componentes fortemente ligadas do exemplo 1 podem ser observados na figura 1 (c).

Como visto em cima e pelo facto de os grafos serem esparsos, optou-se por fazer a representação dos grafos através de listas de adjacências ao invés da representação por matriz de adjacências (figura 1 (b)). Por conseguinte, para um grafo $G=(V,E)$, fez-se um arranjo de $|V|$ listas, uma para cada vértice em $V$. Para cada $v$ $\in$ $V$, a lista de $Adj[u]$ contém ponteiros para todos os vértices $w$ tais que existe uma aresta $(v,w)$ $\in$ $E$, ou seja, $Adj[v]$ consiste em todos os vértices adjacentes a $v$ em $G$.

\begin{figure}[hbt]
        \centering
        \begin{subfigure}[b]{0.28\textwidth}
                \includegraphics[width=\textwidth]{graph.jpg}
                \caption{}
                \label{fig:graph}
        \end{subfigure}%
        ~ %add desired spacing between images, e. g. ~, \quad, \qquad etc.
          %(or a blank line to force the subfigure onto a new line)
        \begin{subfigure}[b]{0.28\textwidth}
                \includegraphics[width=\textwidth]{graphSCC.jpg}
                \caption{}
                \label{fig:graphSCC}
        \end{subfigure}
        ~ %add desired spacing between images, e. g. ~, \quad, \qquad etc.
          %(or a blank line to force the subfigure onto a new line)
        \begin{subfigure}[b]{0.28\textwidth}
                \includegraphics[width=\textwidth]{lista.jpeg}
                \caption{}
                \label{fig:lista}
        \end{subfigure}
        \caption{Exemplo 1 do enunciado. (a) Representação do grafo orientado $G$ com oito vértices e 10 arcos. (b) Representação dos componentes fortemente ligados de $G$. (c) Representação de $G$ como uma lista de adjacências.}\label{fig:grafos}
\end{figure}


Assim, ao nível do código, foram criadas três estruturas essenciais.
Uma estrutura chamada \emph{graph} que contém o vector de listas de adjacência, o número de vértices $N$ (ou pessoas, se virmos da perspectiva do problema), o número de arcos do grafo $P$ ou de partilhas entre as pessoas e, por fim, um vector com o número da componente fortemente ligada. 
O vector de listas de adjacências de um grafo tem uma lista ligada para cada vértice do grafo. A lista do vértice $v$ contém todos os adjacentes de $v$.
A lista de adjacência de um vértice $v$ é composta por nós do tipo \emph{node}. Cada nó da lista corresponde a um arco e contém um adjacente $w$ de $v$ e o endereço do nó seguinte da lista. Um \emph{link} é um ponteiro para um node.
Uma segunda estrutura essencial no programa, \emph{scc\_result}, com o objectivo de reunir a informação resultante do algoritmo de Tarjan, contém o número de SCC, um vector de vectores com os SCC e outro vector com o tamanho de cada SCC.

\section*{Análise teórica}

Na modelação deste problema usou-se um algoritmo recursivo de tempo linear, ou seja, de tempo $O(V+E)$, para encontrar componentes fortemente ligados de um grafo dirigido $G = (V,E)$, chamado algoritmo de Tarjan. Como se pode ver pelo pseudo-código em baixo, este é um algoritmo que faz uso do algoritmo de busca em profundidade de grafos (DFS). Resumindo, a inicialização dá-se em tempo O($V$), as chamadas à função DFSscc igualmente em O($V$), e, por fim as listas de adjacências são analisadas apenas uma vez, logo, O(E). 

\begin{algorithm}
\caption{Algoritmo de Tarjan}
\begin{algorithmic}[1]
\Procedure {SCC\_Tarjan}{$G$}
    \State visited $=0$
    \State L $=0$
    \ForAll {vertex $u \in V[G)]$}
      \State $d[v] = \infty$
   \EndFor
   \ForAll {vertex $v \in V[G]$}
        \If{$d[v] = \infty$}
           \State DFSscc[$v$]
        \EndIf
   \EndFor
\EndProcedure
\Statex        
        
\Procedure {DFSscc}{$G$,$v$}
    \State d[$v]=$low[$v$]=visited
    \State visited = visited + 1
    \State Push(L,$v$)
    \ForAll {$v \in Adj[v]$}
        \If {$d[w] = \infty$}
           \State DFSscc($v$)
        \EndIf
        
        \If {low[$w$] $<$ low[$v$]}
            \State low[$v$] = low[$w$]
        \EndIf
    \EndFor
    
    
    \If {d[$v$] = low[$v$]}
        \Repeat
             \State $w$ = Pop(L)
        \Until{$v=w$}
    \EndIf
    
\EndProcedure
\end{algorithmic}
\end{algorithm}

Podia-se também ter optado pelo algoritmo de Kosaraju, no entanto, este tinha à partida a desvantagem aparente de ter que realizar duas buscas em profundidade (DFS), o que o tornou logo menos atractivo em termos de eficiência, em comparação com o de Tarjan.

\section*{Resultados da avaliação experimental}

Após completa a codificação, a solução foi submetida a diferentes testes. Numa primeira fase, foram feitos os testes disponibilizados pelos docentes, bem como os dois exemplos do enunciado. Nesta fase, a solução parecia ter um comportamento aceitável, pois as saídas estavam todas correctos. No entanto, estes testes não era suficientemente exaustivos e não testavam todas as hipóteses de entradas. Ao submeter o ficheiro no \emph{Mooshak}, o algoritmo foi submetido a testes mais rigorosos. Das primeiras vezes que foi submetido ocorria um \emph{Time Limit Exceeded} em três testes (10, 12, 15). Posteriormente, melhorou-se o nosso algoritmo mudando os acessos à lista de adjacências e optou-se por marcar os vértices para indicar a qual componente fortemente ligado eles pertencem, verificação essa que demora tempo constante.

No entanto, depois dos devidos ajustes, submeteu-se uma solução optimizada que passou com sucesso aos 16 testes do sistema.

\begin{thebibliography}{9}

\bibitem{lamport94}
  Thomas H. Cormen,
  \emph{Introduction to Algorithms}.
  Charles E. Leiserson, Ronald L. Rivest and Clifford Stein,
  Third Edition,
  September 2009.

\end{thebibliography}
\end{document}
